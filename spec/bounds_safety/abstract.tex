\begin{abstract}
Systems programmers need a systems programming language that
detects or prevents common programming errors involving pointers. This
would improve the reliability and security of systems software, as well
the productivity of systems programmers. This design note describes
Checked C, an extended version of C that provides bounds-checking for
pointers and arrays.

Checked C relies on static and dynamic checking to enforce bounds
safety. It adds new pointer types and array types that are
bounds-checked, yet layout-compatible with existing pointer and array
types. Programmers control the placement of bounds information in data
structures and the flow of bounds information through programs. Static
checking enforces the integrity of the bounds information and allows the
eliding of some dynamic checking. Dynamic checking enforces the
integrity of memory accesses at runtime when static checking cannot.
Checked C is a backwards-compatible extension: existing C programs work
``as is''. Programmers incrementally opt-in to bounds checking, while
maintaining binary compatibility. Checked C builds on extensions to C
proposed by the CCured and Deputy projects.
\end{abstract}