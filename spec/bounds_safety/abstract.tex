% !Tex root = checkedc.tex

\parbox{5.5in}{
Existing software written in C is prone to low-level programming
errors that cause security vulnerabilities and reliability
problems.  This specification describes Checked C, a backwards-compatible
extension to C that provides an opt-in way for C programmers to add
bounds checking and improve the type-safety of existing C code, 
eliminating classes of low-level programming errors.
}

\vspace{11pt}

\parbox{5.5in}{
Checked C adds new pointer types and array types that are
null-checked and bounds-checked. The new pointer types
are layout-compatible with existing pointer
types, making Checked C code binary-compatible with existing C code.
Pointer bounds are described using annotations on variable
declarations that use existing program variables and data.
Bounds are checked dynamically at runtime before memory accesses.
Compiler optimizations eliminate unnecessary runtime bounds checking.
Static checking ensures the correctness of the bounds annotations.}

\vspace{11pt}

\parbox{5.5in}{Checked C also adds checked scopes. Checked scopes
provide guarantees about the bounds-safety and type-safety of code
declaration within them. In checked
scopes, only new bounds-checked and null-checked pointer types
can be used.  Type casts are also restricted to ensure type
safety.  Casts between void pointers and pointers to
 data that contains pointers are not allowed.}

\vspace{11pt}
\parbox{5.5in}{Checked C adds
generic functions and structures and existential types.
This lets most void pointer uses be replaced with type-safe code
instead.  Generic type variables are treated as incomplete types.
The restricts the use of generic type variables so that only 
single copies of functions are needed at runtime.
}



