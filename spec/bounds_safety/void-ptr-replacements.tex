% !Tex root = checkedc.tex

\chapter{Checked language extensions for replacing uses of void pointers}

\section{Introduction}

Pointers to void are well-known to be sources of memory corruption in C
programs. They are cast implicitly to and from other pointer types with no checking.
This can lead to type confusion problems, where pointers to objects of one type are mistakenly
assumed to be pointers to objects of different types.  Storing through the pointers
can corrupt memory directly and indirectly.

Pointers to void are used widely in C for {\em untyped} programming that bypasses
static checking.  Uses include:
\begin{itemize}
\item As handles that hide details of API implementations.  An
API may provide a handle that is typed as \uncheckedptrvoid\ for 
users the API, with the API implementation using a pointer to an actual
type.
\item As pointers to arrays of bytes.  This is encouraged by APIs such as \lstinline+memcpy+.
\item In generic data structure or functions. For example, 
a program may have a data structure and functions for operating on lists. 
The list may  have elements typed as \uncheckedptrvoid\ to allow the code
to be used for many different types of list elements, 
even though in practice only one type of data may be stored in a list instance. 
\item For registering callback functions that are to be provided with 
user-supplied data at the callback.  Callback functions can take \uncheckedptrvoid{} arguments and 
the user-suppled data can be cast to \uncheckedptrvoid{}.
\item To provide a union of pointer types, without changing data representation
\end{itemize}

This chapter describes language extensions for replacing uses of \uncheckedptrvoid{}
with type-checked code that cannot cause type confusion.   These extensions
are restricted so that code remains data-layout compatible and binary-code
compatible with existing code.  We propose adding:
\begin{itemize}
\item Generic structs, functions, and type definitions:
a struct or function is generic if it can be re-used for many types.  For example,
we can re-use a generic structure for a  \lstinline+List+ for many different types of list
elements.  Generic functions can implement generic list operations such as \lstinline+Append+.

\item Opaque types: these are incomplete types with enough information to represent
the types, but nothing else.  They can be copied around and used, but the internal
details remain unknown in the scope of the declaration.  No conversions to and 
from opaque types are allowed.

\item Hidden types: these generalize opaque types to handle callbacks involving
user-supplied data.  Hidden types allow us to package up a callback function and data and
say that they use  some type \lstinline+T+ whose details are hidden from the code that does
the callback.   Enough details about \lstinline+T+ are available that the callback can be 
implemented in a type-safe fashion efficently.  Hidden types hide details of the type 
of user-supplied data, instead of erasing it like \uncheckedptrvoid{} does.
\end{itemize}

Generic structs and functions can be combined with the bounds-checking 
to provide type-safe interfaces of functions that operate on arrays of bytes.
Informally, bounds checking provides a way to guarantee that entire objects are handled,
avoiding corruptions caused by operating only on partial objects.  Generic types
can also be combined with opaque types to replace existing unsafe APIs with polymorphic
equivalents.

Generic structs, generic functions, and hidden types use type variables to represent
unknown or partially-known types. Values with types given by type variables must
have sizes in bytes equal to or smaller than the size in bytes of \lstinline+void *+. 
Given a  type variable \lstinline+T+, \lstinline+sizeof(T) == sizeof(void *)+.  We 
restrict the set of types to which generic structs and generic functions can be applied
to  types whose sizes in bytes are less than or equal to the size in bytes 
of \lstinline+void *+. For hidden types, only types whose sizes  in bytes are less than 
or equal to the size in bytes of \lstinline+void *+ can be hidden.

The C Definition is silent on calling conventions.  This means that C implementations
are free to have calling conventions that depend on the type of a parameter.
For example, floating-point arguments may be passed differently than integer 
arguments.
This can create problems for generic functions.  A generic
function could be required to receive arguments differently or return
values differently depending on their their types.  One solution for
sovling this problem is to require cloning of code.  However, we wish to avoid
that because it goes against C philosophy of providing low-level control.

We assume instead that all C implementations that support Checked C pass and
return all pointers the same way.  We also assume that integers whose sizes are less
than or equal to pointer sizes are passed and returned the same way as pointers.  This
is a pragmatic choice already supported by most C implementations.  If a generic
function has an argument type or return type that has a type given by a type
variable, it is implementation-defined whether that generic function applied to
non-pointer and non-integer types can be converted to a non-generic function
ponter type. If the calling conventions line up between the actual type for
the type variable and pointers and integers, an implementation may allow this.

We restrict the uses of type variables in Checked C to provide
backward compatibility:
\begin{itemize}
\item Type variables cannot be used directly as the types of variables or members.
This is so that the layout of stack frames and structures is known at compile-time 
and efficent code can be generated.  For example, given a type variable \lstinline+T+, 
a variable of \lstinline+T *+ can be declared, not a variable of \lstinline+T+.  
\item If \lstinline+sizeof(T)+ is needed at runtime for pointer arithmetic,
the programmer must supply its value.  A generic
function that does pointer arithmetic on a \lstinline+T *+ must be passed 
\lstinline+sizeof(T)+ as an additional argument.  We add a way to declare that
a variable or member holds \lstinline+sizeof(T)+.
This is in keeping with the low-level nature of C.
In some approaches to generic code, the compiler adds runtime information behind the scenes 
as additional data that is passed to a function. 
\end{itemize}
Code cloning is {\em not} needed to
implement generic functions and structs in Checked C, unlike C++ templates.
The Checked C extension of C continues to be a low-level language close
to the hardware.

\section{Language Extensions}
This section explains the extensions to Checked C for 
generic functions, generic structures, opaque types, and hidden types
and provides examples that illustrate how the extensions can be used to
replace uses of \uncheckedptrvoid{}.

\subsection{Generic functions}
\label{sec:functions}
We start with \lstinline+bsearch+, a C standard library function for binary
searching an array of elements of some type \var{T}.  It
takes a key that is a pointer to T, the array of elements, the number of elements of
the array, the size of T, and a comparison function.  We provide the original declaration
of bsearch and the version modified to use a generic type.  In the modified version,
\lstinline+for_any(T)+ make \lstinline+bsearch+ a generic function.  It means that
\lstinline+bsearch+ works for  any type \lstinline+T+.
\begin{lstlisting}
// Original version
void *bsearch(const void *key, const void *base, size_t nmemb, size_t size,
              int ((*compar)(const void *, const void *)));

// Generic version (not correct)
for_any(T) T *bsearch(const T *key, const T *base, size_t nmemb, size_t size
                      int ((*compar)(const T *, const T *)));
\end{lstlisting}
To use the generic version of \lstinline+bsearch+, one instantiates \lstinline+bsearch+ to
a specific type.  For programmer familiarity, we use the C++ syntax for template
instantiations.
\begin{lstlisting}
int arr[] = { 0, 1, 2, 3, 5}
int k = 3;
int cmp(int *a, int *b);
bsearch<int>(&k, arr, sizeof(arr), sizeof(int), cmp);
\end{lstlisting}

It is possible to misuse the generic version of \lstinline+bsearch+ and cause
an out-of-bounds memory access: \lstinline+base+ might not point to a sufficently
large array or \lstinline+compar+ might treat its arguments as pointers to arrays.
This can be addressed by adding bounds checking.  Bounds-checking brings
an interesting problem to light: \lstinline+size+  must match the size of \lstinline+T+.

One might think the solution is that \lstinline+bsearch+ should not take 
\lstinline+size+ as an argument.  The implementation of bsearch could just 
use \lstinline+sizeof(T)+. However, this information is not known at compile-time
within the implementation of \lstinline+bsearch+.   \lstinline+sizeof(T)+
has to be passed in as an argument.  To indicate that \lstinline+size+ must
hold \lstinline+sizeof(T)+, we introduce a constraint on the parameter \lstinline+sizes+:
\begin{lstlisting}
// Generic version (correct)
for_any(T) ptr<T> bsearch(ptr<const T> key,
                          array_ptr<const T> base : byte_count(nmemb * size),
                          size_t nmemb, size_t size : sizeof(T),
                          ptr<int (ptr<const T>, ptr<const T>) compar);
\end{lstlisting}
To keep the example simple, we ignore that this function actually needs a 
bounds-safe interface.  We describe the bounds-safe interface in 
Section~\ref{sec:bounds-safe-interfaces}.

Other functions from the C standard library could be given generic types too:
\begin{lstlisting}
// Generic versions (not correct)
for_any(T) T *malloc(size_t size);

for_any(T) T *calloc(size_t nmemb, size_t size : sizeof(T))

for_any(T) void free(T *pointer)

for_any(T) T *memcpy(T * restrict dest, const T *src, size_t n);
\end{lstlisting}
These functions need bounds checking also.  A programmer
could pass the wrong value for \lstinline+size+ to \lstinline+malloc+
ore \lstinline+calloc+, for example.  Here are
versions with bounds checking (ignoring for now the need for bounds-safe interfaces):
\begin{lstlisting}
// Generic versions (correct)
for_any(T) array_ptr<T> malloc(size_t size) : byte_count(size);

for_any(T) array_ptr<T> calloc(size_t nmemb, size_t size : 
                           sizeof(T)) : byte_count(nmemb * size);

for_any(T) void free(array_ptr<T> pointer : count(1));

for_any(T) array_ptr<T> memcpy(restrict array_ptr<T> dest : byte_count(n),
                               array_ptr<const T> src, byte_count(n),
                               size_t n where n % sizeof(T) == 0) :
                           byte_count(n);
\end{lstlisting}
In the case of \lstinline+malloc+, if the size is not a multiple of the size of \lstinline+T+,
only enough space for part of an object of type \lstinline+T+ will be allocated.  With the 
bounds-safe interface, though, the program will only be able to read or write the 
partially allocated space.   Providing a type-safe interface to \lstinline+memcpy+
is more challenging.   It is incorrect to copy only part of an element of T.
\lstinline+T+ could be struct that contains a pointer.  Copying only a few bytes
of the pointer could result in an invalid pointer.
This is handled in the bounds-safe interface by requiring that the
size be a multiple of the size of \lstinline+T+.

\subsection{Generic structures}
We use the example of a generic \lstinline+List+ structure.   In the case of structs,
the \forany{} clause comes after the tag name of the structure.  
Section~\ref{section:foranyalternatives} explains why the \forany{} clauses 
are placed differently.
\begin{lstlisting}
struct List for_any(T) { 
   T *elem;
   List<T> *next;
}
\end{lstlisting}
A function could take a pointer to a list of T and compute its length:
\begin{lstlisting}
for_any(T) len(List<T> *head) {
  int count = 0;
  while (head != null) {
     count++;
     head = head->next;
  }
  return count;
}
\end{lstlisting}
Within the declaration of \lstinline+List+, we do not allow polymorphic recursion.

\subsection{Opaque types}

\subsection{Hidden types}

\subsection{Bounds-safe interfaces}
\label{sec:bounds-safe-interfaces}
Checked C has the concept of bounds-safe interfaces, which allow programmers
to describe the behavior of existing functions with respect to bounds.  Bounds-safe
interfaces allow us to check that existing functions are used properly in 
checked code, while providing backward-compatibility for existing unchecked code.
This section show how bounds-safe interfaces can be extended to describe
existing functions that are actually generic.  It extends the examples
in Section~\ref{sec:functions}.

Here is the existing bounds-safe interface for \lstinline+bsearch+. 
The bounds declaration on \uncheckedptrvoid{} results in a parameter being
treated as having type \arrayptrvoid{} in checked code:
\begin{lstlisting} 
// Current Checked C version declaration
void *bsearch(const void *key : byte_count(size),
              const void *base : byte_count(nmemb * size),
              size_t nmemb, size_t size,
              int ((*compar)(const void *, const void *)) :
                itype(ptr<int(ptr<const void>, ptr<const void>)>)) :
                byte_count(size);
\end{lstlisting}
Here is a version of \lstinline+bsearch+ with a generic bounds-safe interface:
\begin{lstlisting}
// Generic Checked C version
itype(for_any(T)) 
void *bsearch(const void *key : itype(ptr<T>),
              const void *base : itype(array_ptr<T>) && byte_count(nmemb * size)),
              size_t nmemb, size_t size,
              int ((*compar)(const void *, const void *)) :
              itype(ptr<int (ptr<const T>, ptr<const T>)>)) : ptr<T>;
\end{lstlisting}
The \lstinline+itype(for_any(T))+ indicates that the bounds-safe interface for
the function is generic.  The type variable \lstinline+T+ is available in the
bounds-safe interface declarations for parameters and return values.  The parameter 
\lstinline+base+ has both a new type and a bounds declaration.  This is indicated
by combining an \lstinline+itype+ and a bounds expression using \lstinline+&&+.

Here are bounds safe-interfaces for other C standard library
functions:
\begin{lstlisting}
itype(for_any(T)) void *malloc(size_t) : 
                    itype(array_ptr<T>) && byte_count(size)

itype(for_any(T)) void *calloc(size_t nmemb, size_t size : sizeof(T)) : 
                    itype(array_ptr<T>) && byte_count(nmemb * size);

itype(for_any(T)) void free(void *pointer itype(array_ptr<T>) && count(1));

itype(for_any(T)) void *memcpy(void * restrict dest : 
                                 itype(array_ptr<T>) && byte_count(n),
                               const void *T src : 
                                 itype(array_ptr<T>) && byte_count(n), 
                               size_t n where n % sizeof(T) == 0) :
                           itype(array_ptr<T>) && byte_count(n);
\end{lstlisting}

\section{Lexical and grammar changes}
 We introduce two new keywords:
\begin{lstlisting}
_For_any  _Exists
\end{lstlisting}
It is desirable to have all-lowercase versions of the
keywords for readability and ease of typing. The C
preprocessor is used to provide these. A standard header
file \keyword{stdcheckedc.h} has macros that map the 
lowercase versions of keywords to the actual keywords.
Programs that do not have identifiers that conflict with the
lowercase versions of the keywords can include it.
Throught this document, we use the shorter and easier-to-read
lowercase versions of the keywords.

The grammar from the C11 specification \cite{ISO2011} is extended to allow
generic functions:
\begin{tabbing}
\var{declarat}\=\var{ion:}\\
\>\var{declaration-specifiers} \var{init-declarator-list} \texttt{;} \\
\>\ldots{} \\
\\
\var{function-definition:}\\
\>\var{declaration-specifiers} \var{declarator}  
  \var{declaration-list}\textsubscript{opt} \var{compound-statement}\\
\\
\var{declaration-specifiers:}\\
\>\var{for-any-specifier} \var{declaration-specifiers}\textsubscript{opt} \\
\>\ldots{} \\
\\
\var{for-any-specifier:}\\
\>\texttt{\_For\_any (} \var{type-variable-list} \texttt{)} \\
\\
\var{type-variable-list:} \\
\>\var{type-variable} \\
\>\var{type-variable} \texttt{,} \var{type-variable-list}\\
\\
\var{type-variable:} \\
\>\var{identifier} \\
\\
\\
\var{type-specifier:} \\
\>\var{type-variable} \\
\>\ldots{} \\
\end{tabbing}
At most one \var{for-any-specifier} may occur in the list of declaration specifiers
for a declaration or function definition.
The \var{for-any-specifier} introduces a list of type variables into scope.  
The type variables are available in any following \var{declaration-specifiers} that are part
of the declaration.  The scope of the type variables extends to the end of
the declaration or function definition.



