
 \chapter{Interoperation between safe and unsafe code}\label{interoperation-between-safe-and-unsafe-code}

  \begin{enumerate}
  \def\labelenumii{\arabic{enumii}.}
  \item ~
    \subsection{\texorpdfstring{\protect\hypertarget{ux5fToc422906963}{}{\protect\hypertarget{ux5fToc424307687}{}{\protect\hypertarget{ux5fToc426641110}{}{\protect\hypertarget{ux5fToc435434969}{}{\protect\hypertarget{ux5fToc437460796}{}{\protect\hypertarget{ux5fRef438042394}{}{\protect\hypertarget{ux5fToc440445477}{}{\protect\hypertarget{ux5fToc440449259}{}{\protect\hypertarget{ux5fToc440551909}{}{}}}}}}}}}Conversions
    between pointers to objects of different
    types}{Conversions between pointers to objects of different types}}\label{conversions-between-pointers-to-objects-of-different-types}
  \end{enumerate}

Conversions from a pointer to one type to a pointer to a different type
introduce two issues. First, there is type safety. Given a pointer to S
that has been converted to be a pointer to T, is it valid to treat the
memory pointed to by the ponter as being an object of type T instead of
type S? Second, there is bounds safety. Given the pointer, what range of
memory can be accessed validly using that pointer? This section focus on
bounds safety.

Type safety is not addressed by this design note. Of course, violating
type safety can lead to violations of bounds safety. This can happen
when there is a conversion between a safe pointer to an object of
structure type that contains a bounds-checked member and a safe pointer
to an object of another type. A programmer can use the pointer to the
other type to modify the bounds-checked member or its bounds in an
inconsistent fashion. For now, it is the programmer's responsibility to
update bounds-checked members and their bounds properly when using a
safe pointer that results from such a conversion. Conversions between
safe pointers to integral types, floating-point types, or structure that
contain only integral types or floating-point types cannot lead to
violations of bounds safety by themselves.

Three new operators are introduced for converting between the different
kinds of pointer types:
\textbf{bounds\_cast\textless{}}\emph{T}\textbf{\textgreater{}},
\textbf{core\_bounds\_cast\textless{}}\emph{T}\textbf{\textgreater{}},
and \textbf{assume\_bounds\_cast\textless{}}T\textbf{\textgreater{}}.
The syntax of the operators is similar to the syntax of C++ type
conversion operators. \emph{T} is the destination type to which a source
value is being converted. The new operators take one to three arguments,
where the first argument is the pointer being converted and additional
arguments describe the desired new bounds.

The meanings of the operators when T is a complete type are described
first.

\begin{quote}
\emph{Here are the rules for when a cast operator (T) can be applied to
an expression e with type S:}
\end{quote}

\begin{itemize}
\item
  \emph{T is an array\_ptr: always. The bounds are inferred from the
  expression following the rules in Section . Note that the result may
  be an array\_ptr with bounds(none).}
\item
  \emph{T is a ptr: sometimes. The bounds are inferred as well. The
  resulting bounds must be large enough to hold at least a single
  element of type T. This implies the bounds cannot be bounds(none).}
\item
  \emph{T is an unsafe ptr and S is safe pointer: sometimes. Like ptr, }

  \begin{itemize}
  \item
    \emph{If S is ptr type and size of the of the referent }

    \begin{enumerate}
    \def\labelenumi{\arabic{enumi}.}
    \item ~
      \subsubsection{\texorpdfstring{\protect\hypertarget{ux5fToc437460797}{}{\protect\hypertarget{ux5fToc440445478}{}{\protect\hypertarget{ux5fToc440449260}{}{\protect\hypertarget{ux5fToc440551910}{}{}}}}Dynamic
      checking using
      bounds\_cast\textless{}\emph{T}\textgreater{}}{Dynamic checking using bounds\_cast\textless{}T\textgreater{}}}\label{dynamic-checking-using-boundsux5fcastt}
    \end{enumerate}
  \end{itemize}
\end{itemize}

\textbf{bounds\_cast\textless{}}\emph{T}\textbf{\textgreater{}} uses
dynamic checks to enforce the preservation of bounds safety. \emph{T}
must be a pointer type and is the desired destination type. The operator
takes 1 to 3 arguments, depending on the kind of conversion being done:

\begin{itemize}
\item
  \textbf{bounds\_cast\textless{}}\emph{T}\textbf{\textgreater{}(}\emph{e1},
  \emph{e2}\textbf{)} converts \emph{e1} to an \textbf{array\_ptr} or
  \textbf{array\_view} type with bounds
  \textbf{count(}\emph{e2}\textbf{)}.
\item
  \textbf{bounds\_cast\textless{}}\emph{T}\textbf{\textgreater{}(}\emph{e1}\textbf{,}
  \emph{e2}\textbf{,} \emph{e3}\textbf{)} converts \emph{e1} to an
  \textbf{array\_ptr} or \textbf{array\_view} type with bounds
  \textbf{bounds(}\emph{e2}\textbf{,} \emph{e3}\textbf{)}.
\item
  \textbf{bounds\_cast\textless{}}\emph{T}\textbf{\textgreater{}(}\emph{e1}\textbf{)}
  converts \emph{e1} to either a \textbf{ptr} or * type.
\end{itemize}

It is a compile-time error if the bounds of e1 are
\textbf{bounds(none)}. No runtime checks are done if the bounds of e1
are \textbf{bounds(any)}. If the bounds of e1 are
\textbf{bounds(}\emph{lb}\textbf{,} \emph{ub}\textbf{)}, the following
runtime checks are done:

\begin{itemize}
\item
  \textbf{bounds\_cast\textless{}}\emph{T}\textbf{\textgreater{}(}\emph{e1},
  \emph{e2}\textbf{)}: if \emph{e1} \textbf{!= 0}, check that \emph{lb}
  \textbf{\textless{}=} \emph{e1} \textbf{\&\&} \emph{e1} \textbf{+
  sizeof(referent-type(}\emph{T}\textbf{)) *} e2 \textbf{\textless{}=}
  \emph{ub}.
\item
  \textbf{bounds\_cast\textless{}}\emph{T}\textbf{\textgreater{}(}\emph{e1}\textbf{,}
  \emph{e2}\textbf{,} \emph{e3}\textbf{)}: if \emph{e1} \textbf{!= 0},
  check that \emph{lb} \textbf{\textless{}=} \emph{e2} \textbf{\&\&}
  \emph{e3} \textbf{\textless{}=} \emph{ub}. Also check that relative
  alignment constraints for \emph{e1}, \emph{e2}, and \emph{e3} are met.
\item
  \textbf{bounds\_cast\textless{}}\emph{T}\textbf{\textgreater{}(}\emph{e1}\textbf{)}:
  check that there is room for least one element of \emph{T}. This is
  the same as doing the check for
  \textbf{bounds\_cast\textless{}}T\textbf{\textgreater{}(}e1,
  1\textbf{)}: if \emph{e1} \textbf{!=} \textbf{0}, check that \emph{lb}
  \textbf{\textless{}=} \emph{e1} \textbf{\&\&} \emph{e1} \textbf{+}
  \textbf{sizeof(referent-type(}\emph{T}\textbf{) \textless{}=}
  \emph{ub}.

  \begin{enumerate}
  \def\labelenumi{\arabic{enumi}.}
  \item ~
    \subsubsection{\texorpdfstring{\protect\hypertarget{ux5fToc437460798}{}{\protect\hypertarget{ux5fToc440445479}{}{\protect\hypertarget{ux5fToc440449261}{}{\protect\hypertarget{ux5fToc440551911}{}{}}}}Correctness
    of bounds\_cast\textless{}\emph{T}\textgreater{}
    conversions}{Correctness of bounds\_cast\textless{}T\textgreater{} conversions}}\label{correctness-of-boundsux5fcastt-conversions}
  \end{enumerate}
\end{itemize}

In this section, the correctness of the bounds\_cast conversions is
discussed. Readers interested in just using the operations may safely
skip this discussion.

We discuss only
\textbf{bounds\_cast\textless{}T\textgreater{}(}\emph{e1}\textbf{,}
\emph{e2}\textbf{,} \emph{e3)} because the other forms are special cases
of this operation. From Section 4, recall the meaning of a bounds
expression \textbf{bounds(}\emph{lb}\textbf{,} \emph{ub}\textbf{)} for
an expression \emph{e1} at runtime. Let the runtime values of \emph{e1},
\emph{lb}, and \emph{ub} be \emph{e1v}, \emph{lbv}, and \emph{ubv},
respectively. The value \emph{e1v} will be \textbf{0} or have been
derived via a sequence of operations from a pointer to some object
\emph{obj} with \textbf{bounds(}\emph{low}\textbf{,}
\emph{high}\textbf{)}. The following statement will be true: \emph{e1v}
\textbf{== 0 \textbar{}\textbar{} (}\emph{low} \textbf{\textless{}=}
\emph{lbv} \textbf{\&\&} ubv \textbf{\textless{}=}
\emph{high}\textbf{)}.

To ensure \textbf{bounds(}\emph{e2}\textbf{,} \emph{e3}\textbf{)} is
valid for \emph{e1}, we need to show the following:

\begin{quote}
Let \emph{e2v} and \emph{e3v} be the runtime values for \emph{e2} and
\emph{e3} respectively. Then \emph{e1v} \textbf{== NULL
\textbar{}\textbar{} (}\emph{low} \textbf{\textless{}=} \emph{e2v}
\textbf{\&\&} e3v \textbf{\textless{}=} \emph{high}\textbf{)}.
\end{quote}

Given \emph{e1v} \textbf{== 0 \textbar{}\textbar{} (}\emph{low}
\textbf{\textless{}=} \emph{lbv} \textbf{\&\&} ubv \textbf{\textless{}=}
\emph{high}\textbf{)}, the runtime check that if \emph{e1v} \textbf{!=}
\emph{0}, \emph{lbv} \textbf{\textless{}=} \emph{e2v} \textbf{\&\&}
e\emph{3v} \textbf{\textless{}=} \emph{ubv} implies \emph{e1v}
\textbf{== 0 \textbar{}\textbar{} (}\emph{low} \textbf{\textless{}=}
\emph{e2v} \textbf{\&\&} e3v \textbf{\textless{}=}
\emph{high}\textbf{).}

\subsubsection{\texorpdfstring{\protect\hypertarget{ux5fToc437460799}{}{\protect\hypertarget{ux5fToc440445480}{}{\protect\hypertarget{ux5fToc440449262}{}{\protect\hypertarget{ux5fToc440551912}{}{}}}}Static
checking use
core\_bounds\_cast\textless{}\emph{T}\textgreater{}}{Static checking use core\_bounds\_cast\textless{}T\textgreater{}}}\label{static-checking-use-coreux5fboundsux5fcastt}

\textbf{core\_bounds\_cast\textless{}}\emph{T}\textbf{\textgreater{}}
uses static checking to enforce the preservation of bounds safety:

\begin{itemize}
\item
  \textbf{core\_bounds\_cast\textless{}}\emph{T}\textbf{\textgreater{}(}\emph{e1},
  \emph{e2}\textbf{)} converts \emph{e1} to an \textbf{array\_ptr} or
  \textbf{array\_view} type with bounds
  \textbf{count(}\emph{e2}\textbf{)}.
\item
  \textbf{core\_bounds\_cast\textless{}}\emph{T}\textbf{\textgreater{}(}\emph{e1}\textbf{,}
  \emph{e2}\textbf{,} \emph{e3}\textbf{)} converts \emph{e1} to an
  \textbf{array\_ptr} or \textbf{array\_view} type with bounds
  \textbf{bounds(}\emph{e2}\textbf{,} \emph{e3}\textbf{)}.
\item
  \textbf{core\_bounds\_cast\textless{}}\emph{T}\textbf{\textgreater{}(}\emph{e1}\textbf{)}
  converts \emph{e1} to either a \textbf{ptr} or * type.
\end{itemize}

Like \textbf{bounds\_cast\textless{}}\emph{T}\textbf{\textgreater{}}, it
is a compile-time error if the bounds of e1 are \textbf{bounds(none)}.
There is no static checking if the bounds of e1 are
\textbf{bounds(any)}. If the bounds are \textbf{bounds(}lb\textbf{,}
ub\textbf{)}, the following must be provable via static checking:

\begin{itemize}
\item
  \textbf{core\_bounds\_cast\textless{}}\emph{T}\textbf{\textgreater{}(}\emph{e1},
  \emph{e2}\textbf{)}: under the assumption that \emph{e1} != 0, lb
  \textless{}= \emph{e1} \&\& \emph{e1} + sizeof(referent-type(T)) * e2
  \textless{}= ub.
\item
  \textbf{core\_bounds\_cast\textless{}}\emph{T}\textbf{\textgreater{}(}\emph{e1}\textbf{,}
  \emph{e2}\textbf{,} \emph{e3}\textbf{)}: under the assumption that
  \emph{e1} \textbf{!= 0}, \emph{lb} \textbf{\textless{}=} \emph{e2}
  \textbf{\&\&} \emph{e3} \textbf{\textless{}=} \emph{ub}. Also show the
  relative alignment constraints for \emph{e1}, \emph{e2}, and \emph{e3}
  are met: \textbf{(}e3 \textbf{--} \emph{e2}\textbf{) \%
  sizeof(referent-type(}\emph{T}\textbf{)) == 0 \&\& (e2 -- e1) \%
  sizeof(referent-type(}\emph{T}\textbf{)}), where arithmetic is done in
  bytes.
\item
  \textbf{core\_bounds\_cast\textless{}}\emph{T}\textbf{\textgreater{}(}\emph{e1}\textbf{)}
  converts \emph{e1} to either a \textbf{ptr} or * type. For this, check
  that
  \textbf{core\_bounds\_cast\textless{}}\emph{T}\textbf{\textgreater{}(}\emph{e1},
  \emph{1}\textbf{)} holds.
\end{itemize}

\textbf{core\_bounds\_cast} can be used when programmers do not want to
introduce the possibility of dynamic runtime failures due to bounds
casts. It is recommended for conversions between pointers to
constant-length arrays, which succeed at runtime using
\textbf{core\_bounds\_cast} if and only if they succeed at run-time
using \textbf{bounds\_cast}. For example, converting an array\_ptr of 12
characters to an array\_ptr of 4 32-bit integers should always succeed
at compile-time.

\subsubsection{\texorpdfstring{\protect\hypertarget{ux5fToc437460800}{}{\protect\hypertarget{ux5fToc440445481}{}{\protect\hypertarget{ux5fToc440449263}{}{\protect\hypertarget{ux5fToc440551913}{}{}}}}Conversions
without checking using
assume\_bounds\_cast\textless{}\emph{T}\textgreater{}}{Conversions without checking using assume\_bounds\_cast\textless{}T\textgreater{}}}\label{conversions-without-checking-using-assumeux5fboundsux5fcastt}

\textbf{assume\_bounds\_cast\textless{}}T\textbf{\textgreater{}}
converts unsafe pointers to safe pointers. The declared bounds are
trusted without verification, although relative alignment is checked if
necessary. The source expression must have an unsafe pointer type.

\begin{itemize}
\item
  \textbf{assume\_bounds\_cast\textless{}}\emph{T}\textbf{\textgreater{}(}\emph{e1},
  \emph{e2}\textbf{)} converts \emph{e1} to an \textbf{array\_ptr} or
  \textbf{array\_view} type with count(e2)
\item
  \textbf{assume\_bounds\_cast\textless{}}\emph{T}\textbf{\textgreater{}(}\emph{e1}\textbf{,}
  \emph{e2}\textbf{,} \emph{e3}\textbf{)} converts \emph{e1} to an
  \textbf{array\_ptr} or \textbf{array\_view} type with bounds(e2, e3).
  There is a runtime check that the relative alignment requirements for
  T are satisfied by \emph{e1,} \emph{e2} and \emph{e3}. \emph{e2} and
  \emph{e3} can have pointer type.
\item
  \textbf{assume\_bounds\_cast\textless{}}\emph{T}\textbf{\textgreater{}(}\emph{e1}\textbf{)}
  converts \emph{e1} to a \textbf{ptr} type or T \textbf{*}.
\end{itemize}

A subtle point about
\textbf{bounds\_cast\textless{}}\emph{T}\textbf{\textgreater{}} and
\textbf{core\_bounds\_cast\textless{}}\emph{T}\textbf{\textgreater{}}
are that they allow \emph{\emph{bounds-safe casts}} of an expression
\emph{e} of unsafe pointer type to an \textbf{array\_view} or
\textbf{array\_ptr} type. This is provided that the bounds for \emph{e}
can be determined. Only a few kinds of expressions with unsafe pointer
types have bounds that can be determined: address-of expressions
involving variables and uses of array variables, where the array
variable is converted to a pointer type.

\subsubsection{\texorpdfstring{\protect\hypertarget{ux5fToc437460801}{}{\protect\hypertarget{ux5fToc440445482}{}{\protect\hypertarget{ux5fToc440449264}{}{\protect\hypertarget{ux5fToc440551914}{}{}}}}Summary
of conversion
operators}{Summary of conversion operators}}\label{summary-of-conversion-operators}

The following table summarizes the operations for converting between
pointers to objects of different types, provided that S and T are
complete types:

\begin{longtable}[c]{@{}lllll@{}}
\toprule
To

From & array\_ptr\textless{}T\textgreater{} &
array\_view\textless{}T\textgreater{} & ptr\textless{}T\textgreater{} &
T *\tabularnewline
\midrule
\endhead
array\_ptr\textless{}S\textgreater{} &
bounds\_cast\textless{}array\_ptr\textless{}T\textgreater{}\textgreater{}(e1,
e2)

bounds\_cast\textless{}array\_ptr\textless{}T\textgreater{}\textgreater{}(e1,
e2, e3)

core\_bounds\_cast\textless{}array\_ptr\textless{}T\textgreater{}\textgreater{}(e1,
e2)

core\_bounds\_cast\textless{}array\_ptr\textless{}T\textgreater{}\textgreater{}(e1,
e2, e3) &
bounds\_cast\textless{}array\_view\textless{}T\textgreater{}\textgreater{}(e1,
e2)

bounds\_cast\textless{}array\_view\textless{}T\textgreater{}\textgreater{}(e1,
e2, e3)

core\_bounds\_cast\textless{}array\_view\textless{}T\textgreater{}\textgreater{}(e1,
e2)

core\_bounds\_cast\textless{}array\_view\textless{}T\textgreater{}\textgreater{}(e1,
e2, e3) &
bounds\_cast\textless{}ptr\textless{}T\textgreater{}\textgreater{}(e1)

core\_bounds\_cast\textless{}ptr\textless{}T\textgreater{}\textgreater{}(e1)
& bounds\_cast\textless{}T *\textgreater{}(e1)

If S==T, \&e1{[}i{]} can be used too.

assume\_bounds\_cast\textless{}T *\textgreater{}(e1)\tabularnewline
array\_view\textless{}S\textgreater{} & Same as above & Same as above &
Same as above & Same as above.\tabularnewline
ptr\textless{}S\textgreater{} & Same as above & Same as above. & Same as
above.

For bounds\_cast, it is a compile-time error if sizeof(S) \textless{}
sizeof(T). This is guaranteed to fail at runtime. &
bounds\_cast\textless{}T *\textgreater{}(e1)

assume\_bounds\_cast\textless{}T *\textgreater{}(e1)\tabularnewline
S * & If e1 has known bounds, same as above.

Otherwise:

assume\_bounds\_cast\textless{}array\_ptr\textless{}T\textgreater{}\textgreater{}(e1,
e2)

assume\_bounds\_cast\textless{}array\_ptr\textless{}T\textgreater{}\textgreater{}(e1,
e2, e3) & If e1 has known bounds, same as above.

Otherwise:

assume\_bounds\_cast\textless{}array\_view\textless{}T\textgreater{}\textgreater{}(e1,
e2)

assume\_bound\_cast\textless{}array\_view\textless{}T\textgreater{}\textgreater{}(e1,
e2, e3). & If e1 has known bounds, same as above.

Otherwise:

assume\_bounds\_cast\textless{}T\textgreater{}(e1), & C cast operation
of the form (T *) e1\tabularnewline
\bottomrule
\end{longtable}

\protect\hypertarget{ux5fToc424307715}{}{}

\subsection{\texorpdfstring{\protect\hypertarget{ux5fToc437460802}{}{\protect\hypertarget{ux5fToc440445483}{}{\protect\hypertarget{ux5fToc440449265}{}{\protect\hypertarget{ux5fToc440551915}{}{\protect\hypertarget{ux5fToc426641112}{}{\protect\hypertarget{ux5fToc435434971}{}{}}}}}}Bounds-safe
conversions for pointers to incomplete types, including pointers to
void}{Bounds-safe conversions for pointers to incomplete types, including pointers to void}}\label{bounds-safe-conversions-for-pointers-to-incomplete-types-including-pointers-to-void}

The prior section , the source and destination pointer types in a
bounds\_cast or core\_bounds\_cast operation must be complete types or
have a referent type of void. However, there is one important exception:
ptr\textless{}T\textgreater{} can always be safely converted to T *,
even if T is incomplete. This follows from treating sizeof(T) as a
symbolic constant in the rules for computing bounds and the runtime
checking rules for bounds\_cast and core\_bounds\_cast.

This exception is useful when converting code incrementally from unsafe
pointer types to safe pointers. One may have converted the body of a
function to use safe pointers, but not yet have converted some functions
that it calls.

C allows pointer types to be freely converted to void * types and void *
types to be freely converted back to pointer types. For safe pointer
types, any kind of safe pointer type can be freely converted to the same
kind of pointer type with a referent type of void. However, explicit
casts must be used when converting from any kind of safe pointer type
with a referent type of void:

\begin{longtable}[c]{@{}lllll@{}}
\toprule
To From & T * & ptr\textless{}T\textgreater{} &
array\_view\textless{}T\textgreater{} &
array\_ptr\textless{}T\textgreater{}\tabularnewline
\midrule
\endhead
void * & Follow existing C rules & ptr\_cast\textless{}T\textgreater{},
if bounds are known.

assume\_ptr\_cast\textless{}T\textgreater{}, otherwise &
bounds\_cast\textless{}array\_view\textless{}T\textgreater{}\textgreater{},
if bounds are known:

- check relative alignment

assume\_bounds\_cast\textless{}array\_view\textless{}T\textgreater{}\textgreater{}(e1,
e2): e2 is length. No relative alignment check needed

assume\_bound\_cast\textless{}array\_view\textless{}T\textgreater{}\textgreater{}(e1,
e2, e3). e2/e3 are lower/upper bounds.

- check relative alignment &
bounds\_cast\textless{}array\_ptr\textless{}T\textgreater{}\textgreater{}:
- has bounds(none) if bounds are unknown.

- check relative alignment if bounds are known.

assume\_bounds\_cast\textless{}array\_ptr\textless{}T\textgreater{}\textgreater{}(e1,
e2): e2 is length, no relative alignment check needed

assume\_bound\_cast\textless{}array\_ptr\textless{}T\textgreater{}\textgreater{}(e1,
e2, e3): e2/e3 are lower/upper bounds

- check relative alignment\tabularnewline
ptr\textless{}void\textgreater{} & Only the following unsafe operator is
available.

assume\_unsafe\_ptr\_cast: no check &
ptr\_cast\textless{}T\textgreater{}

\emph{\textbf{TODO: need unsafe operator}} & Not allowed & Not
allowed\tabularnewline
view\textless{}void\textgreater{} & unsafe\_ptr\_cast: check that memory
accessible by pointer dereference is in bounds

assume\_unsafe\_ptr\_cast: no check &
ptr\_cast\textless{}T\textgreater{}

check that memory accessible by pointer dereference is in bounds &
bounds\_cast\textless{}array\_view\textless{}T\textgreater{}\textgreater{}:

- check relative alignment &
bounds\_cast\textless{}array\_ptr\textless{}T\textgreater{}\textgreater{}:

- check relative alignment\tabularnewline
array\_ptr\textless{}void\textgreater{} & unsafe\_ptr\_cast: check that
memory accessible by pointer dereference is in bounds

assume\_unsafe\_ptr\_cast: no check &
ptr\_cast\textless{}T\textgreater{}

check that memory accessible by pointer dereference is in bounds &
bounds\_cast\textless{}array\_view\textless{}T\textgreater{}\textgreater{}:

- check relative alignment &
bounds\_cast\textless{}array\_ptr\textless{}T\textgreater{}\textgreater{}:

- check relative alignment\tabularnewline
\bottomrule
\end{longtable}

\subsection{\texorpdfstring{\protect\hypertarget{ux5fToc426641113}{}{\protect\hypertarget{ux5fToc435434972}{}{\protect\hypertarget{ux5fToc437460803}{}{\protect\hypertarget{ux5fRef437611522}{}{\protect\hypertarget{ux5fToc440445484}{}{\protect\hypertarget{ux5fToc440449266}{}{\protect\hypertarget{ux5fToc440551916}{}{}}}}}}}Bounds-safe
interfaces to existing unsafe
functions}{Bounds-safe interfaces to existing unsafe functions}}\label{bounds-safe-interfaces-to-existing-unsafe-functions}

The new pointer types capture specific properties of pointers. One would
like to update existing C code to use these new pointer types. However,
this will not be possible when backward compatibility requirements
exist. Consider C runtime functions or OS APIs. It may be feasible to
modify the header files for the runtime function or OS API. It may be
impossible to require all uses of these functions or APIs be updated.
Code written by 3\textsuperscript{rd} parties may use these APIs and it
is not reasonable for an existing shipping OS to require all
3\textsuperscript{rd} parties update their existing code to use the new
types.

Consider what would happen if the signature for memcpy were updated to
use \textbf{array\_ptr}. The function

void *memcpy(void *dest, const void *src, size\_t count);

becomes

void *memcpy(array\_ptr\textless{}void\textgreater{} dest,
array\_ptr\textless{}const void\textgreater{} src, size\_t count);

This, of course, breaks every piece of existing code that uses memcpy.
The code will no longer compile. C does not have method overloading, so
one cannot simply define multiple overloaded versions of memcpy. That
would also be duplicative and potentially increase program sizes.

The reverse problem also exists: suppose the signature for memcpy is not
updated. Then every ``safe'' method that calls memcpy would need to cast
the arguments to unsafe pointer types!

Given that we may not be able to change the pointer types of existing
APIs, we need to adopt an approach that supports backwards
compatibility, enables new safe code to be written easily, and maintains
the safety of new code.

We address this by:

\begin{enumerate}
\def\labelenumi{\arabic{enumi}.}
\item
  Allowing programmers to declare bounds-safe interfaces to code and
  data structures that use unsafe pointers. A bounds-safe interface for
  a function, for example, describes bounds for unsafe pointer
  parameters.
\item
  In checked scopes, where only safe pointers are allowed, unsafe
  pointer types in bounds-safe interfaces are reinterpreted as safe
  pointer types. This makes the code in those scopes straightforward to
  understand: only safe pointers are used, all memory accesses are
  bounds checked or in bounds, and bounds-safe interfaces are trusted
  and respected.
\item
  In unchecked scopes, where safe and unsafe pointer types are allowed,
  we separate type checking concerns from bounds checking concerns. We
  allow implicit coercions at uses of bounds-safe interfaces between
  safe pointer types and unsafe pointer types. We also allow implicit
  coercions from unsafe pointer types to safe pointer types, subject to
  the limitation the bounds checking must also succeed.
\end{enumerate}

Functions that have parameters with unsafe pointer types or that return
values with unsafe pointer types can have bounds declared for the unsafe
pointer parameters or the unsafe pointer return value. Bounds must be
declared for all parameters with unsafe pointer types and the return
value, if it has an unsafe pointer type. In the case where a parameter
has \textbf{ptr} type, this can be declared specially.

Here is the bounds-safe interface for memcpy:

void *memcpy(void *dest : count(len), const void *src : count(len),
size\_t len)

where return\_value : bounds(dest, dest + count)

The correctness of bounds information is enforced at compile-time when
memcpy is passed safe pointer arguments. It is not enforced when memcpy
is passed unsafe pointer arguments.

Similarly, for data structures, members with unsafe pointer types can
have bounds declared. If bounds are declared for one member of a
structure with an unsafe pointer type, they must be declared for all
members with unsafe pointer types.

Here are the bounds for a structure that is a counted buffer of
characters.

struct S \{

char *arr : count(len);

int len;

\}

We may have a method that takes a counted buffer of characters and
counts the number of instances of a specific character. The \textbf{ptr}
declaration can be used to declare an unsafe pointer to a singleton
object of a type:

int count\_char(S *str where ptr, char arg);

It is very important to understand that the \emph{semantics of unsafe
pointers does not change even when bounds are declared for the
pointers}. The declared bounds are used only when checking safe code
that calls unsafe functions or uses unsafe data structures. Unsafe
pointer dereferences do not have bounds checks added to them. A method
that declares a bounds-safe interface and whose body consists of only
unsafe code is compiled as though the bounds-safe interface has been
stripped from its source code.

\subsection{\texorpdfstring{\protect\hypertarget{ux5fRef424639711}{}{\protect\hypertarget{ux5fToc426641114}{}{\protect\hypertarget{ux5fToc435434973}{}{\protect\hypertarget{ux5fToc437460804}{}{\protect\hypertarget{ux5fToc440445485}{}{\protect\hypertarget{ux5fToc440449267}{}{\protect\hypertarget{ux5fToc440551917}{}{}}}}}}}implicit
conversions}{implicit conversions}}\label{implicit-conversions}

In unchecked scopes, we allow implicit coercions from safe pointer types
to the corresponding unsafe pointer types in the following cases. In
each case, one or more types is reinterpreted and type checking proceeds
with the reinterpreted types:

\begin{itemize}
\item
  Function call:

  \begin{itemize}
  \item
    If the function being called has a bounds-safe interface for unsafe
    pointer type arguments and an argument expression has a safe pointer
    type, then the type of each argument of a safe pointer type will be
    reinterpreted as the corresponding unsafe pointer type.
  \end{itemize}
\item
  Field assignment: if the left hand side has an unsafe pointer type,
  the right-hand side has a safe pointer type, and the left-hand side is
  a member of a type with a bounds-safe interface, then the right-hand
  side type will be reinterpreted as the corresponding unsafe pointer
  type.
\end{itemize}

We allow coercions from unsafe pointer types to the corresponding safe
pointer types in the following cases:

\begin{itemize}
\item
  Function call:

  \begin{itemize}
  \item
    If a function being called has a parameter of a safe pointer type
    and the type of an argument is an unsafe pointer, then the type of
    the argument will be reinterpreted as the corresponding safe pointer
    type.
  \end{itemize}
\item
  Assignment: if the left hand side has a safe pointer type to T, the
  right-hand side has an unsafe pointer type to T, then the right-hand
  side type will be reinterpreted as the corresponding safe pointer
  type.
\end{itemize}

The rules for checking bounds are applied separately. These rules are
extended to deal with bounds-safe interface functions as follows.

\begin{enumerate}
\def\labelenumi{\arabic{enumi}.}
\item
  Respecting bounds interfaces:

  \begin{enumerate}
  \def\labelenumii{\alph{enumii}.}
  \item
    If function call has arguments with safe pointer types that are
    converted implicitly to unsafe pointer types, then the bounds for
    the arguments must satisfy the bounds requirements for the unsafe
    pointer parameters. In addition,

    \begin{enumerate}
    \def\labelenumiii{\roman{enumiii}.}
    \item
      If an argument corresponds to a parameter that has a bounds
      declaration, the argument will be checked to see if it is in
      bounds following the bounds declaration.
    \item
      If an argument corresponds to a parameter that has \textbf{ptr}
      declaration, the argument will be checked to see if it is in
      bounds of its computed bounds.
    \end{enumerate}
  \item
    TBD: field assignment to a member of a type with a bound-safe
    interface (the issue is that we never infer when a bounds is valid).
  \end{enumerate}
\item
  Trusting bounds interfaces:

  \begin{enumerate}
  \def\labelenumii{\alph{enumii}.}
  \item
    If a function call returns an unsafe pointer and the function has a
    bounds-safe interface, the bounds for the return value will be
    computed the same as if the function returned an \textbf{array\_ptr}
    type.
  \item
    If a member access returns an unsafe pointer and the member access
    is to a type with a bounds-safe interface, the member access will
    have the bounds declared for it by the member bounds declared in the
    type.
  \end{enumerate}
\end{enumerate}

Here are some examples:

void f()

\{

int a;

// the type of \&a will be int *. This will be implicitly converted to

// ptr\textless{}int\textgreater{}.

ptr\textless{}int\textgreater{} pa : bounds(\&a, \&a + 1) = \&a;

int arr{[}5{]}{[}5{]};

// the type of arr will be int{[}5{]} *. This will be implicitly
converted to

// ptr\textless{}int{[}5{]}\textgreater{}

ptr\textless{}int{[}5{]}\textgreater{} parr : bounds(arr, arr + 1) =
arr;

...

\}

void copy(array\_ptr\textless{}int\textgreater{} dest : count(len),

array\_ptr\textless{}int\textgreater{} src : count(len), int len)

\{

// dest, src will be converted to a void *

// The checking of bounds information succeeds at compile time.

// The runtime bounds checks that dest and src are in range are

// guaranteed to succeed

memcpy(dest, src, len);

\}

f(S s)

\{

int len = s.len;

array\_ptr\textless{}char\textgreater{} sp : count(len) = s.arr;

\ldots{}

if (len \textgreater{} 0) \{

sp{[}0{]} = `a';

\}

\}

The following example will fail at compile time. The bounds for sp will
be computed as \textbf{bounds(none)}, which clearly does not match
\textbf{count(sp, sp + 1)}.

char *random();

f() \{

array\_ptr\textless{}char\textgreater{} sp : count(1) = random();

\}
 
 