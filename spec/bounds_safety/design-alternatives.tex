% !Tex root = checkedc.tex

\chapter{Design choices considered but not chosen}
\label{chapter:design-alternatives}

\section{Support for \arrayptr\ and \arrayview\ pointers that are not relatively aligned}\label{support-for-arrayux5fptr-and-arrayux5fview-pointers-that-are-not-relatively-aligned}

We considered supporting the case of \arrayptr\ or
\arrayview\ pointers where pointers and their bounds are not
guaranteed to be relatively aligned. A programmer might use such
pointers in coding patterns where code strides through an array of bytes
interpreted as an array of integers or structures. Typically, the code
checks that it has not gone too far before accessing memory. For
example, given an array of bytes, a programmer may wish to replace byte
operations with aligned 32-bit operations because they are more
efficient. Extra bytes at the beginning or end of an array would be
handled using special-case code.

When bounds and pointers are not relatively aligned, the bounds checks
against the upper bound for a pointer requires extra instructions. Given
a pointer p, *p accesses the memory from p to p + sizeof(T) -- 1. Given
an upper bound \var{ub}, the upper bounds check for \var{p} becomes p
+ sizeof(T) -- 1 \textless{} ub. Note that the computation of p +
sizeof(T) -- 1 would also need an overflow check, so it would typically
result in several extra instructions, not just an extra addition.

The upper bounds check would be harder to optimize. Most programmers
would write code that strides through an array using comparison that
\var{p} \textless{} \var{ub}. The comparison \var{p} \textless{}
\var{ub} does not imply \var{p} + sizeof(T) -- 1 \textless{}
\var{ub}, so it would not be sufficient for a compiler to optimize away
the upper bounds check. A compiler would have to know that the pointer
and bounds are relatively aligned in order to eliminate the upper bounds
check. It would be hard for a compiler to prove this because that would
require interprocedural or whole-program analysis.

If all \arrayptr\ operations with bounds checks used the more
general bounds check in order to support pointers that are not
relatively aligned, there would be significantly more bounds checks in
optimized code and the checks would also be more costly.

We considered the alternative of introducing a new pointer modifier,
such as ``ragged'', that could be used with \arrayptr\
or \arrayview\ types to indicate that pointers and bounds may
not be relatively aligned. A type with the ragged modifier would have
the more costly upper bounds check. Because ragged and regular
\arrayptr\ pointers would have different bounds checking
sequences, casting away the unaligned modifier would compromise the
integrity of bounds checking. It would be illegal to cast a ragged
\arrayptr\ or \arrayview\ pointer to an
\arrayptr\ or \arrayview\ with no ragged modifier.

We chose not to add the ragged modifier to the language because it would
be adding a new type modifier to support a relatively rare situation.
This would add language complexity for little-to-no benefit in practice.
For example, code that would use this feature typically also has logic
to detect or handle leftover bytes. This logic can be adapted to compute
bounds that are properly relatively aligned.

\section{Allowing pointer variables to be assigned values with no relationship to their bounds}\label{allowing-pointer-variables-to-be-assigned-values-with-no-relationship-to-their-bounds}

We considered allowing pointer variables to be assigned pointer values
not derived in some way from the object with which their bounds are
associated. The idea would be to avoid unnecessary restrictions on
operations involving pointers and give pointers more leeway to encode
information by modifying bits in pointers.

In this approach, the meaning of a bounds expression would be defined
differently than that given in Section 4 . The meaning would be th
following. Given an expression \var{e} with a bounds expression
\texttt{bounds(}\var{lb}\texttt{,} \var{ub}\texttt{)}, let the runtime
values of \var{e}, \var{lb}, and \var{ub} be \var{ev}, \var{lbv},
and \var{ubv}, respectively. If \var{ev} is not null, there will
always exist some object at runtime with the bounds (\var{low},
\var{high}) such that \var{low} \texttt{\textless{}=} \var{lbv} \&\&
\var{ubv} \texttt{\textless{}=} \var{high}. In other words, the
requirement is that expression bounds are always a subrange of the range
of memory for some valid object. This is provided that the value of the
expression with which those bounds are associated is non-null.

The problem with this approach is that it has unexpected consequences
for the bounds that are allowed to be declared for pointer variables.
Any valid pointer bounds could be declared for a variable because there
is no longer a requirement that a pointer stored in the variable is
derived from a pointer to the object associated with the bounds. The
following example would be valid:

\begin{verbatim}
array_ptr<int> x : count(5) = malloc(sizeof(int) * 5);
array_ptr<int> y : bounds(x, x + 5) = malloc(sizeof(int) * 2);
\end{verbatim}

Because the bounds are disassociated from the actual pointer values,
there is much more potential for errors to be made by programmers that
are only detected at runtime. This approach was not pursued further for
this reason.

\section{Address-of and array-to-pointer conversion always produce safe pointer types}\label{address-of-and-array-to-pointer-conversion-always-produce-safe-pointer-types}

We considered a designed where the address-of operator (\&) and
array-to-pointer type conversion always produced safe pointers. To
preserve compatibility with existing C code, we introduce implicit
conversions from safe pointers to unsafe pointers. We found that we were
not able to preserve backwards compatibility for the address-of operator
and that implicit array-to-pointer conversions required bounds checking.

\subsection{Proposed rules for an address-of operator that produces safe pointer type.}\label{proposed-rules-for-an-address-of-operator-that-produces-safe-pointer-type.}

The address-of operator (\texttt{\&}) applied to an lvalue expression of
type \var{T} would produce a value of type
\ptrT.

Existing C code expects the address-of operator to produce a \var{T} *.
To allow most code to compile without changes, we add an implicit cast
rule: \ptrT can be cast
implicitly to a \var{T} * in those situations where a \var{T}
\texttt{*} type is expected, except for pointer arithmetic operators
that add or subtract a pointer and an integer. Those situations include
pointer assignment, arguments to function calls, return statements, and
conditional expressions. In all these situations a \var{T} \texttt{*}
type must have been declared explicitly in the code already, so this
implicit cast does not introduce unsafe pointer types where none existed
before.

Pointer arithmetic operators are excluded to avoid the silent
introduction of unsafe pointer types and to preserve the value of having
the \ptrT type. If
there were always an implicit cast from \ptrT to \var{T} \texttt{*},
then any expression that uses pointer arithmetic could do pointer
arithmetic on \ptrT
values.

Code takes the address of an array element and immediately does pointer
arithmetic will still fail to type check, introducing a potential
backward compatibility issue:
\begin{verbatim}
f()
{
    int a[10];
    int *x = &a[0] + 5; // &a[0] has type ptr<T>.  Pointer arithmetic is not allowed
    ...
}
\end{verbatim}

We expect this kind of code to be rare because the succinct style is to
use \texttt{a} instead of \texttt{\&a[0]}, but is nonetheless a
possibility, so this proposal still violates the principle of not
changing the meaning of existing C code.

\begin{verbatim}
f()
{
    int a[10];
    int *x = ((int *) &a[0]) + 5; // redundant but OK under old rule
    …
}
\end{verbatim}

\subsection{Proposed rules for conversion of array types to pointer types}\label{proposed-rules-for-conversion-of-array-types-to-pointer-types}

Array types may be complete or incomplete. A complete array type
specifies the number of elements in the array using a constant
expression. In incomplete array type does not specify the number of
elements in the array. Examples of complete array types are int[10]
and int[10][10]. Examples of incomplete array types are
int[] and int[][10].

If the type of an expression or a subexpression is an ``array of
\var{T}'', the following rules would apply. If the array is a complete
type, the type of the expression is altered to
\arrayptrT . If it is
an incomplete type, the type of the expression is altered to \var{T} *.
This alteration does not happen if the expression is an operand of an
address-of operator, \texttt{++}, \texttt{-\/-}, \texttt{sizeof}, or the
left operand of an assignment operator or the `\texttt{.}' operator.

These rules would have an interesting effect for arrays of complete
types: all array references involving those arrays would be bounds
checked. Any address computations involving those arrays will be checked
for overflow also. Because the existing C language definition leaves
out-of-bounds access of arrays of complete type undefined, as well as
the meaning of overflowing address computations undefined, this is
compatible with the existing C language definition.

However, these rules by themselves are problematic for existing C code.
It is common in C code to use array types interchangeably with pointer
types. The rule that complete array types are converted to
\arrayptr\ types could cause problems for such code

\begin{verbatim}
f(int *arg, int len)
{ 
   ...
}

g() {
   int x[10];
   f(x, 10);
}

h() {
   int x[10];
   int *ptr = x;
   f(ptr, 10);
}
\end{verbatim}

To allow existing code to continue to compile unchanged, we adopt the
rule that an \arrayptrT\ can be
implicitly cast to a \var{T} * in situations where a T * type is
expected. Those situations may include pointer assignment, arguments to
function calls, return statements, and conditional expressions. For
conditional, expressions of the form exp1 \texttt{?} exp2 \texttt{:}
exp3, the implicit coercion occurs when exp2 or exp3 has type T * and
the other expression has type
\arrayptrT. These situations do not
include array references and adding or subtracting a pointer type and an
integer. \arrayptrT\ is an acceptable
type for those operations and a coercion to T * is not needed.

We allow \arrayptr\ values to not be within bounds. Because of
this, any implicit conversion of an \arrayptr\ value with a
bounds to an unsafe pointer type must be bounds checked. Otherwise, it
is easy to write ``safe'' code that creates undetected buffer overruns:

\begin{verbatim}
// f looks safe, but does something bad that is undetected before calling unsafe code
f(array_ptr<int> p where bounds(p) == (p, p + 10))
{
    // first argument implicitly converted to int *
    poke(p + random_large_value(), 31415);  
}

void poke(int *p, int val)
{
    *p val
}
\end{verbatim}

The silent introduction of a bounds check at a call to a method violates
the design principles of control and clarity. The implicit conversion
introduces an invisible failure point in a program where one does not
otherwise exist. Pointer arithmetic is not normally bounds checked, so
it is not expected fail.

\section{Alternate definitions of the semantics of bounds declarations}\label{alternate-definitions-of-the-semantics-of-bounds-declarations}

There are a variety of different possible definitions of the semantics
of bounds declarations. A bounds declaration has the form:

\var{bounds-decl:}

\begin{quote}
\var{x} \texttt{:} \var{bounds-exp}
\end{quote}

\var{bounds-exp:}

\begin{quote}
\texttt{count(}\var{non-modifying-exp}\texttt{)}

\texttt{bounds(}\var{non-modifying-exp}\texttt{,}
\var{non-modifying-exp}\texttt{)}

\boundsany

\boundsnone
\end{quote}

It may be attached to declarators or an assignment statement:

\var{init-declarator} :

\begin{quote}
\var{declarator inline-bounds-specifier\textsubscript{opt}
where-clause\textsubscript{opt}}

\var{declarator inline-bounds-specifier\textsubscript{opt}
where-clause\textsubscript{opt}} \texttt{=} \var{initializer
where-clause\textsubscript{opt}}

\ldots{}
\end{quote}

\var{parameter-declaration} :

\begin{quote}
\var{declaration-specifiers declarator}
\var{inline-bounds-specifier\textsubscript{opt}
where-clause\textsubscript{opt}}
\end{quote}

\var{inline-bounds-specifier:}

\begin{quote}
\texttt{:} \var{bounds-exp}
\end{quote}

\var{where-clause}:

\begin{quote}
\keyword{where} \var{facts}
\end{quote}

\var{expression-statement}:

\begin{quote}
\var{expression\textsubscript{opt}
where-clause\textsubscript{opt}}\texttt{;}
\end{quote}

The information in the bounds declaration is used at pointer
dereferences involving either (1) the variable or (2) pointers
constructed from the value of the variable.

One question is when the values of the bounds expressions in a bounds
declaration are computed. One possibility is to compute them eagerly at
the point of the bounds declaration. This make sense in that expressions
are evaluated at the point where they occur in C programs. However, a
problem with evaluating bounds expressions early is that they may fault,
even though their value is not needed.

Consider code that calls malloc, which returns null if memory cannot be
allocated. The code checks that malloc succeeded before using the
result.
\begin{verbatim}
array_ptr<int> result = malloc(size) where result : bounds (result, result + size)
if (result != NULL) {
      ... *result = ...
}
\end{verbatim}

In this case, the bounds expression result + size will fault when null
is returned. One way to think about the problem is that it is acceptable
for null pointers to have meaningless bounds expressions. They cannot be
used to access memory, after all. Of course, this is a problem if the
bounds expression causes a runtime fault. Another way to think about
this is that the problem is that the bounds expression is being
evaluated whether it is needed or not. The evaluation should be deferred
until it is definitely required.

There are currently several different definitions of extent under
consideration. The definitions differ on two axes:

\begin{enumerate}
\item
  The set of program points to which a declaration applies. The
  possibilities include:

  \begin{enumerate}
  \item
    The set of statements up to the first assignment to any variable
    used in the bounds declaration. We will name this set of statements
    the ``direct'' set
  \item
    All statements after the declaration where the variables involved in
    the declaration in scope and a new declaration has not superseded
    this definition. We will name this set the ``always'' set.
  \item
    Various sets of program points in between the two extremes of (a)
    and (b). For the sake of discussion, we will name this set the
    ``intermediate'' set.

    \begin{enumerate}
    \item
      The set of statements up to the first non-invertible assignment to
      any variable used in the declaration (an assignment is
      non-invertible if the old value of the variable cannot be
      calculated from the new variable).
    \item
      The set of statements up to the first assignment to any variables
      used in the bounds expressions in the declaration.
    \item
      The set of statements up to the first assignment where the bounds
      declaration can no longer be proved to be true (where the
      right-hand side of the bounds declaration may no longer be a valid
      range.
    \end{enumerate}
  \end{enumerate}
\item
  When the bounds expressions in a bounds declaration are evaluated. The
  possibilities being considered are:

  \begin{enumerate}
  \item
    By-value (eagerly at the point of declaration)
  \item
    By-reference (at the point of use of the variable with the variable
    declaration).
  \end{enumerate}
\end{enumerate}

Here is the matrix of different permutations:

\begin{longtable}[c]{@{}llll@{}}
\toprule
Set of program points & By-value & By-reference & General
comments\tabularnewline
\midrule
\endhead
Direct & & & By-value and by-reference evaluation is indistinguishable.
Because there are no assignments, both produce the same
result\tabularnewline
Always & Introduces hidden temporary variables that hold bounds value.
May lead to runtime failures if the lhs variable is assigned a pointer
to an object outside of the declared bounds & Modification of variables
used in bounds expressions requires declaring a new bounds for the lhs
for variable. &\tabularnewline
Intermediate & Introduces hidden temporary variables that hold bounds
value.

May lead to runtime failures if the lhs variable is assigned a pointer
to an object outside of the declared bounds. & &\tabularnewline
General comments & We would need to check that all pointer variables in
bounded expressions are non-null before evaluating bounds at
declaration.

Otherwise evaluation may fail at runtime if a variable on the right-hand
side is null. & &\tabularnewline
\bottomrule
\end{longtable}
